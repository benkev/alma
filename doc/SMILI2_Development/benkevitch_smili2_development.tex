\documentclass[preprint]{aastex}

\begin{document}

\title{Development and Validation of the SMILI2 Input/Output for CASA FITS Files}

\author{L. V. Benkevitch}
\affil{MIT Haystack Observatory, Westford MA}

SMILI2 is a Python package for imaging multidimensional visibility data from radio interferometer arrays. SMILI2 is being developed in the framework of Atacama Large Millimeter/submillimeter Array (ALMA). CASA, the Common Astronomy Software Applications package, is the radio astronomy data processing software being developed and maintained by NRAO, National Radio Astronomical Observatory. CASA can be considered as a successor to the older and mature package AIPS. 

It is common to save, keep, and transfer brightness image data in the FITS format files. However, SMILI2 has its native format for keeping the image data different from the FITS and other standards. In order to import and export external images the SMILI2 Image class has a set of methods to load and save the internal image data in FITS format. Unfortunately, different astronomy software systems have different FITS formats and metadata, so SMILI2 has to have several methods to load/save the FITS files in the most common formats.

This document describes the development and testing of the \verb@load_fits_casa()@ and \verb@to_fits_casa()@ methods to load/save the image FITS files in the format generated by CASA and accepted by CASA. 

\begin{enumerate}
  \item select\_bandpols.py
  		
  \item generate\_pcmt.py
  \item compare\_pcmt.py
  \item hist\_pcmt.py
\end{enumerate}


\section{Introduction}



\section{Appendix}

The software is located at the github repository \\
\verb@https://github.com/benkev/alma@ \\
The user can create a local copy at a disk location using the command \\
\verb%$ git clone git@github.com:benkev/alma.git alma% \\

Currently, the software does not require compilation or installation.



\end{document}