\documentclass[preprint]{aastex}

\begin{document}

\title{Development and Validation of the SMILI2 Input/Output for CASA FITS Files}

\author{L. V. Benkevitch}
\affil{MIT Haystack Observatory, Westford MA}

SMILI2 is a Python package for imaging multidimensional visibility data from radio interferometer arrays. SMILI2 is being developed in the framework of Atacama Large Millimeter/submillimeter Array (ALMA). CASA, the Common Astronomy Software Applications package, is the radio astronomy data processing software being developed and maintained by NRAO, National Radio Astronomical Observatory. CASA can be considered as a successor to the older and mature package AIPS. 

It is common to save, keep, and transfer brightness image data in the FITS format files. However, SMILI2 has its native format for keeping the image data different from the FITS and other standards. In order to import and export external images the SMILI2 Image class has a set of methods to load and save the internal image data in FITS format. Unfortunately, different astronomy software systems have different FITS formats and metadata, so SMILI2 has to have several methods to load/save the FITS files in the most common formats.

This document describes the development and testing of the \verb@load_fits_casa()@ and \verb@to_fits_casa()@ methods to load/save the image FITS files in the format generated by CASA and accepted by CASA. Also, described here are the Python scripts created for the testing: 

\begin{enumerate}
  \item \verb@alma_util.py@
  \item \verb@RoundSpottyDisk_Betelgeuse.py@
  \item \verb@load_save_Betelgeuse.py@
  \item \verb@diff_fits.py@
  \item \verb@sim_observ.py@
  \item \verb@diff_ms.py@
  \item \verb@self_cal_steps.py@
  \item \verb@diff_cal.py@
\end{enumerate}


\section{A Utility Module}

A simple module \verb@alma_util.py@ currently provides the only function, \verb@workdir.py@, returning a string with the ``work" directory path. It helps make scripts running universally on any of the machines used by the SMILI2 developers. The ``work" directory (supposedly containing the ALMA and \verb@smili2_dev@ directories) is different on different servers. On leonid2 and capelin (owned by Lynn Matthews) it is \verb@/data-smili/@. On isco it is \verb@/data-isco/data-smili/@. On my machine it is my home directory, \verb@/home/benkev/@, etc. 

In order to use it in CASA scripts, the user should copy \verb@alma_util.py@ to his or her CASA installation \verb@site-packages@ directory, which is like \\ \\
\verb@casa-release-5.7.2-4.el7/lib/python2.7/site-packages/@.


\section{Creating a FITS Model Image of Betelgeuse}

The script \verb@RoundSpottyDisk_Betelgeuse.py@ creates a circular spotted disk at the Betelgeuse sky coordinates at  \verb@RA=05h55m10.0800s@, \verb@DEC=07d24m25.200s@ (equinox \verb@J2000@), the size equivalent to 80 $\mu$as, which is the size of Betelgeuse found by Lim \emph{et al.} at 46 GHz. The script must be run in CASA: \\ \\
\verb@execfile('RoundSpottyDisk_Betelgeuse.py')@ \\
\\
This command saves the model Betelgeuse image in the file \verb@RoundSpottyDisk_alma.fits@.


\section{Loading and Saving the FITS Image with SMILI2}

In SMILI2 the class \verb@imdata@ provides two methods, \verb@imdata.load_fits_casa()@ and \\ \verb@imdata.to_fits_casa()@ to load and save CASA-generated fits files. The script \\
\verb@load_save_Betelgeuse.py@ located in the top-level SMILI2 directory \verb@smili2_dev@ uses these methods. It loads the file \verb@RoundSpottyDisk_alma.fits@ converting it into the internal SMILI2 format and then converts it back into the FITS format saving it under the name \\
\verb@RoundSpottyDisk_smili.fits@. The script \verb@imdata.load_fits_casa()@ should be run outside of CASA, for example, it runs in IPython as \\ \\
\verb@%run load_save_Betelgeuse.py @


\section{Testing the FITS Image Datacubes Identity}

In order to prove that SMILI2 saves FITS files in the true CASA format we first tested the identity of the datacubes in FITS files created by CASA, \verb@RoundSpottyDisk_alma.fits@, and created by SMILI2, \verb@RoundSpottyDisk_smili.fits@. The script \verb@diff_fits.py@ reads the data from both FITS files into memory and compares them, printing if they are identical or not. It can be run as \\
\\
\verb@%run diff_fits.py@ \\
\\
After a few debugging iterations, SMILI2 has proved to save the data in the FITS file totally identical to that of the input CASA-generated FITS file.


\section{Simulated Observation of Both FITS Images}

Observing the two FITS images, the original one and its copy saved by SMILI2, must result in two strictly identical measurement sets with the visibility data. CASA provides a task to simulate an observation of a celestial image, \verb@simobserve()@. The script \verb@sim_observ.py@ uses the task to to generate a simulated ALMA measurement set with 2-hour continuous integrations. It also generates one more noisy data set. The noise assumes 10 GHz bandwidth and dual polarization. 

The script \verb@sim_observ.py@ can be used on the two FITS images being compared, \\ \verb@RoundSpottyDisk_alma.fits@ and \verb@RoundSpottyDisk_smili.fits@. \\
For this it should be run in CASA twice with the command \\
\\
\verb@execfile('sim_observ.py')@ \\
\\
Before each run, the string variables projname and projdir should be set. The former will be a part of the output filenames, while the latter will be the directory to save the output files. In our experiments the following names have been used: \\ \\
\verb@projdir = 'alma_orig'; projname = 'alma'@ -- to observe \verb@RoundSpottyDisk_alma.fits@ \\
and \\
\verb@projdir = 'smili_orig'; projname = 'smili'@ -- to observe \verb@RoundSpottyDisk_smili.fits@. \\

To keep the originals \verb@alma_orig@ and \verb@smili_orig@ safe, they were copied into the directories \verb@alma@ and \verb@smili@.



\section{Testing the CASA Measurement Sets' Visibility Data Identity}

Comparison of the resulting measurement sets in the two directories, \verb@alma_orig@ and \verb@smili_orig@, was made with the script \verb@diff_ms.py@: \\
\\
\verb@execfile('diff_ms.py')@ \\
\\
It prints the maximum difference between the absolute values and phases of the complex visibilities and plots them versus time. 

First runs showed identity in absolute values of the visibilities before and after SMILI2, but differences in their phases. Examination of the FITS header keyword set copied into the SMILI2 metadata showed that it lacked the following data: \\
\\
\verb@EQUINOX = 2000.0@ and \verb@RADESYS = 'FK5'@. \\
\\
Here the value of \verb@EQUINOX@ is set to the \verb@epoch J2000@, the time 2000/1/1 12:00:00 in the Gregorian calendar of the reference point for celestial coordinates, the vernal equinox, which is continuously changing due to the earth precession. The keyword \verb@RADESYS@ provides the celestial coordinate reference frame, here FK5 (The Fifth Fundamental Catalogue). However, CASA by default uses the modern celestial coordinate system, ICRS, which does not refer to the vernal equinox, and its reference axes are fixed in space.  

The corresponding items were introduced in the SMILI2 metadata set as \verb@equinox@ ans \verb@coordsys@. Saving the FITS files with correct \verb@EQUINOX@ and \verb@RADESYS@ values lowered the discrepancies between the observation visibility data.



\section{Testing the Calibration Tables Identity}

The visibility image created in an interferometric observation is converted into the sky map via the 2D Fourier transform with subsequent cleaning based on the CLEAN algorithm. After cleaning, the gains and phases of each of the array antenna amplifiers need calibration. This can be done with a separate sky source or with the observed sky object itself. In the last case the process is called self-calibration. The (self) calibration results in the calibration table containing complex gains for each of the antennas. Usually, several rounds of CLEANing and calibration are required. 

The idea of testing if SMILI2 saves the FITS images correctly is as follows. The measurement sets (visibilities) for the original image and the one saved by SMILI2 are independently cleaned/self-calibrated. The FITS images will be proven truly identical if several rounds of the self-calibration generate identical calibration tables.

The script \verb@self_cal_steps.py@  performs four iterations of the self-calibration saving the calibration tables at each iteration. The script is run from CASA: \\
\\
\verb@execfile('self_cal_steps.py')@
\\ \\
Two string variables parameterize the script execution: \\
\\
\verb@proj:@ the project name, \verb@proj = 'alma'@ or \verb@proj = 'smili'@; \\
\verb@ns:  @ \verb@ns = '_noise'@ for the noisy observation, or \verb@ns = ''@ for the noiseless one.
\\

The script \verb@diff_cal.py@ compares the calibration tables. It plots a histogram of the phase differences between values in the calibration tables being compared for all antennas. Also, it plots the means and the standard deviations of distributions of the phase differences between values in the calibration tables being compared for each antenna. 

The output of \verb@diff_cal.py@ showed discrepancies in the calibration tables, indicating that there are the FITS header keywords not taken into account. The following keywords were found to be instrumental: \\
\\
\verb@LONPOLE = 180.0@ \\                                               
\verb@LATPOLE =   7.407@ \\                                                 
\verb@OBSGEO-X =  2225142.180269@ \\
\verb@OBSGEO-Y = -5440307.370349@ \\
\verb@OBSGEO-Z = -2481029.851874@ \\
\\
Here \verb@LONPOLE@ and \verb@LATPOLE@ are the  native longitude and latitude of the celestial North pole, while \verb@OBSGEO-X@, \verb@OBSGEO-Y@, and \verb@OBSGEO-Z@ are the Cartesian geographic X, Y, and Z coordinates of the observatory. These keywords were included into the SMILI2 metadata set as \verb@lonpole@, \verb@latpole@, \verb@obsgeo-x@, \verb@obsgeo-y@, and \verb@obsgeo-y@, respectively. Now SMILI2 started saving the FITS files with the mentioned values. 

After this improvements, running all over again the scripts \\ \\
\verb@RoundSpottyDisk_Betelgeuse.py@ \\
\verb@load_save_Betelgeuse.py@ \\
\verb@diff_fits.py@ \\
\verb@sim_observ.py@ \\
\verb@diff_ms.py@ \\
\verb@self_cal_steps.py@ \\
\verb@diff_cal.py@ \\
\\
showed completely zero discrepancies at all the stages from image creation to self-calibrated and cleaned observation results. It finally showed that SMILI2 creates FITS files totally identical to the CASA output FITS files.


\section{Appendix}

The testing software is located at the github repository \\
\verb@https://github.com/benkev/alma@ \\
The user can create a local copy at a disk location using the command \\
\verb%$ git clone git@github.com:benkev/alma.git alma% \\

Currently, the software does not require compilation or installation.

\end{document}